\documentclass{article}
\usepackage{ctex}  %加载包,因为我们在用中文写文档,所以必须加载这个包,否则不支持中文
\usepackage{multicol}  %加载包
\usepackage[top=1in, bottom=1in, left=1.25in, right=1.25in]{geometry}  %加载包
\usepackage{lscape}
\usepackage[colorlinks,linkcolor=blue]{hyperref}
\usepackage{amsmath}
\author{VincentZhang}  % 作者
\date{2019.7.30}  %定义时间
\title{流体力学}  %文档标题
\begin{document}

\maketitle
前言,这篇文章是2007年Siggraph的课程文档(Course Note),从头到底很详细的讲解了怎么用计算机模拟流体(烟和水等等)。
原文地址: \url{https://www.cs.ubc.ca/~rbridson/fluidsimulation/fluids_notes.pdf}

\section{第一章 流体力学公式}
流体力学中最重要的公式是Navier-Stokes公式:
\begin{equation}
\frac{\partial{\vec{u}}}{\partial{t}}+\vec{u}\cdot \nabla{\vec{u}}+\frac{1}{\rho}\nabla{p}=\vec{g}+\nu\nabla\cdot\nabla\vec{u} \label{momentumequation}
\end{equation}
\begin{equation}
\nabla\cdot\vec{u}=0 \label{incompressibilitycondition}
\end{equation}

\subsection{符号含义}
其中$\vec{u}$是流体的速度。\par
$\rho$是流体的密度,大家都熟悉,如果是水的话,其值为$1000kg/m^3$. 如果是空气的话,其值为$1.3kg/m^3。$\par
$p$是压力,流体对内对外的压力。\par
$vec{g}$是大家都熟悉的重力加速度,一般性就是中学书上的值$(0,-9.81,0)m/s^2$。从这个向量表示也行,我们这里y轴像上。x-轴,y-轴是水平的。$\vec{g}$有时候大家也会叫他"体作用力"\par
$\nu$是“静态粘度”。顾名思义,这个参数表达的是流体有多粘稠。像糖浆这样的流体这个值很高,像酒精这样的流体这个值很低。学术点说,这个参数反映了流体抗拒形变的程度。一个极端的例子是沥青,网上不是有个段子,有个科学家等了一辈子想看一颗沥青掉下来的过程,最后到死都没有实现。另一个更加极端的例子是玻璃,玻璃其实也是一种流体,有很多几百年老教堂的玻璃窗,下面比上面厚,这就是流体流动的结果,但是大家懂的,要观察玻璃滴下来,估计这个老教授要等到宇宙尽头了。
\subsection{动量方程}
Navier-Stokes里的第一条方程\ref{momentumequation},其实是一条向量方程,另有一个名字叫做"动量方程",所以看这条方程的时候,千万要注意变量上面的箭头。其实这个方程就是变形的牛二方程$\vec{F}=m\vec{a}$。这个方程告诉了我们流体在各种作用力之下是怎么运动的。显然这条方程比较复杂,这一节将把这个方程拆开一项项给大家讲解。后一章节,我们将介绍\ref{incompressibilitycondition},这个方程叫做“不可压缩条件”。\par
现在先假设我们用粒子系统去模拟流体(实际上后面的章节有讲怎么具体去模拟,这里作此假设只是进行理论推导方便)。每一个粒子,都可以被理解为一个小小的"分子团",这个“分子团”的质量为m,体积为V(这点很重要,注意我们不是用单个分子去模拟粒子,而是一个小小的分子团!),速度为$\vec{u}$。 根据牛二定理,只要我们知道了这个“分子团”上面的所有的力,根据$\vec{F}=m\vec{a}$我们就能知道这个“分子团”的加速度了。其中加速度,我们用这种奇怪的方式定义:\begin{equation}
\vec{a}\equiv\frac{D\vec{u}}{Dt}
\end{equation}
这里面大写的D,叫做物质导数或者叫随体导数。如果你能打开这个链接的话,Wiki上有很好的解释\url{https://en.wikipedia.org/wiki/Material_derivative}。百度百科上的解释是这样的:\url{https://baike.baidu.com/item/%E9%9A%8F%E4%BD%93%E5%AF%BC%E6%95%B0} \par
简单来说,随体导数是流体力学中的术语。研究的是流体在某点的力学状态时,常考虑这个点周围很小范围内的物质(也就是我们说的“分子团”)随时间变化的变化率。比如这个”分子团“的体积随时间的变化率,再如”分子团“组成的平面随着时间的变化率等等。\par
用随体导数改写的牛二定理如下:
\begin{equation}
m\frac{D\vec{u}}{Dt}=\vec{F}
\end{equation}
OK,接下来我们就来看下粒子上受到了哪些力。最简单的当然是重力:$m\vec{g}$。但是这肯定不是唯一的力,其他粒子也对这个粒子有作用力。\par
第一种来自其他粒子的力,是压力。高压区域的粒子持续不断的给低压区域的粒子施加压力,比如正是由于有心脏不断跳动,向血液施加压力,血液才能流遍全身)。请注意,我们这里的压力,考虑的是合力,具体这个力是从哪个粒子来的并不重要。大家都挤过地铁吧,想象我们现在站在一节塞满了人的地铁车厢里面,真的是塞得满满当当的。这时候突然你的左手边开门了,瞬间你左边的人都走光了,这时候你会受什么力呢?当然是从右向左的力啦,你右边的人也想下车但是被你挡住了啊,只能推你!\par
具体到流体的粒子,假设流体受到的总压力分布为p(注意,这是一个矢量场,不同的位置有不同的压力值), 那么在某点的压强是多少呢?在该点对压力取梯度就可以了,也就是$-\nabla p$,那么我们这个”分子团“上面受到的总压力是多少呢?理论上讲是对$-\nabla p$在整个粒子团的表面取积分,这么做很难,所以我们一般就简简单单乘以体积进行近似。结果粒子上的总压力为:$-V\nabla p$。负号表示这个力是粒子团抵抗外界压力的力,压力的存在主要是因为粒子团想要维持自己的体积,这一点在后续”不可压缩条件“这部分会有更详细解释。\ par
第二种来自其他粒子的力为粘性力,粘性力的主要作用是抵制流体的形变。粘性力存在的原因应该是流体内部的摩擦力。其效果是让瘤体内的粒子(分子团)的速度尽量跟周围粒子的平均速度一样,也就是说,尽量降低流体内部的速度差。如果你做过图像处理,或者了解热传导方程或扩散方程,那就应该能知道,数学上描述某一个特定点的数量与周围点平均值的差距是拉普拉斯算子$\nabla\cdot\nabla$,这个量在这里是速度,因此这个部分是$\nabla\cdot\nabla\vec{u}$。下一节会更详细解释拉普拉斯算子,别着急。更压力的情况差不多的是,粘性力也需要对整个粒子进行积分。因此我们得到了:$V\nu\nabla\cdot\nabla\vec{u}$,其中$\nu$是粘性系数,第一章解释过。\par
总结下,一个粒子团总共受到三种力,重力、压力、粘性力。把上面的所有项都放在一起,我们就得到了:
\begin{equation}
m\frac{D\vec{u}}{Dt}=m\vec{g}-V\nabla{p}+V\nu\nabla\cdot\nabla\vec{u}
\end{equation}
等式两边除以体积,得到:
\begin{equation}
\rho\frac{D\vec{u}}{Dt}=\rho\vec{g}-\nabla{p}+\nu\nabla\cdot\nabla\vec{u}
\end{equation}
其中$\rho$为流体密度。

\fbox{
\parbox{\textwidth}{
简单解释下拉普拉斯算子$\nabla\cdot\nabla$,详细的解释请参见多维微积分教材 \\
拉普拉斯算子,记作$\nabla\cdot\nabla$、$\nabla^2$或者$\Delta$,为梯度的散度。顾名思义,他的含义是先对函数f(一般至少是二维函数)做梯度$\nabla{f}$,再对这个梯度函数做散度$\nabla\cdot\nabla{f}$。\\
这么说还是很抽象,我们来考虑下二维离散的图片,在某一点的拉普拉斯算子(四联通)可以这样定义:
\begin{equation}
\nabla\cdot\nabla{f}=[f(x+1,y)+f(x-1,y)+f(x,y+1)+f(x,y-1)]-4f(x,y) \label{laplacian}
\end{equation}
如果用卷积核的方式来表示,就是:
$$
\left[
\begin{matrix}
0 & -1 & 0 \\
-1 & 4 & -1 \\
0 & -1 & 0
\end{matrix}
\right]
$$
从\ref{laplacian}可以很明显看出,拉普拉斯算子衡量的是一个点的数值偏离周围点数值平均值的程度。在图像处理上,主要用来做边缘检测。
}
}




\end{document}
