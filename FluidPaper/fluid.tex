\documentclass{article}
\usepackage{ctex}  %加载包,因为我们在用中文写文档,所以必须加载这个包,否则不支持中文
\usepackage{multicol}  %加载包
\usepackage[top=1in, bottom=1in, left=1.25in, right=1.25in]{geometry}  %加载包
\usepackage{lscape}
\usepackage[colorlinks,linkcolor=blue]{hyperref}
\author{VincentZhang}  % 作者
\date{2019.7.30}  %定义时间
\title{流体力学}  %文档标题
\begin{document}

\maketitle
前言,这篇文章是2007年Siggraph的课程文档(Course Note),从头到底很详细的讲解了怎么用计算机模拟流体(烟和水等等)。
原文地址: \url{https://www.cs.ubc.ca/~rbridson/fluidsimulation/fluids_notes.pdf}

\section{第一章 流体力学公式}
流体力学中最重要的公式是Navier-Stokes公式:
\begin{equation}
\frac{\partial{\vec{u}}}{\partial{t}}+\vec{u}\cdot \nabla{\vec{u}}+\frac{1}{\rho}\nabla{p}=\vec{g}+\nu\nabla\cdot\nabla\vec{u} \label{momentumequation}
\end{equation}
\begin{equation}
\nabla\cdot\vec{u}=0 \label{incompressibilitycondition}
\end{equation}

\subsection{符号含义}
其中$\vec{u}$是流体的速度。\par
$\rho$是流体的密度,大家都熟悉,如果是水的话,其值为$1000kg/m^3$. 如果是空气的话,其值为$1.3kg/m^3。$\par
$p$是压力,流体对内对外的压力。\par
$vec{g}$是大家都熟悉的重力加速度,一般性就是中学书上的值$(0,-9.81,0)m/s^2$。从这个向量表示也行,我们这里y轴像上。x-轴,y-轴是水平的。$\vec{g}$有时候大家也会叫他"体作用力"\par
$\nu$是“静态粘度”。顾名思义,这个参数表达的是流体有多粘稠。像糖浆这样的流体这个值很高,像酒精这样的流体这个值很低。学术点说,这个参数反映了流体抗拒形变的程度。一个极端的例子是沥青,网上不是有个段子,有个科学家等了一辈子想看一颗沥青掉下来的过程,最后到死都没有实现。另一个更加极端的例子是玻璃,玻璃其实也是一种流体,有很多几百年老教堂的玻璃窗,下面比上面厚,这就是流体流动的结果,但是大家懂的,要观察玻璃滴下来,估计这个老教授要等到宇宙尽头了。
\subsection{动量方程}
Navier-Stokes里的第一条方程\ref{momentumequation},其实是一条向量方程,另有一个名字叫做"动量方程",所以看这条方程的时候,千万要注意变量上面的箭头。其实这个方程就是变形的牛二方程$\vec{F}=m\vec{a}$。这个方程告诉了我们流体在各种作用力之下是怎么运动的。显然这条方程比较复杂,这一节将把这个方程拆开一项项给大家讲解。后一章节,我们将介绍\ref{incompressibilitycondition},这个方程叫做“不可压缩条件”。\par
现在先假设我们用粒子系统去模拟流体(实际上后面的章节有讲怎么具体去模拟,这里作此假设只是进行理论推导方便)。每一个粒子,都可以被理解为一个小小的"分子团",这个“分子团”的质量为m,体积为V(这点很重要,注意我们不是用单个分子去模拟粒子,而是一个小小的分子团!),速度为$\vec{u}$。 根据牛二定理,只要我们知道了这个“分子团”上面的所有的力,根据$\vec{F}=m\vec{a}$我们就能知道这个“分子团”的加速度了。其中加速度,我们用这种奇怪的方式定义:\begin{equation}
\vec{a}\equiv\frac{D\vec{u}}{Dt}
\end{equation}
这里面大写的D,叫做物质导数或者叫随体导数。如果你能打开这个链接的话,Wiki上有很好的解释\url{https://en.wikipedia.org/wiki/Material_derivative}。百度百科上的解释是这样的:\url{https://baike.baidu.com/item/%E9%9A%8F%E4%BD%93%E5%AF%BC%E6%95%B0} \par
简单来说,随体导数是流体力学中的术语。研究的是流体在某点的力学状态时,常考虑这个点周围很小范围内的物质(也就是我们说的“分子团”)随时间变化的变化率。比如这个”分子团“的体积随时间的变化率,再如”分子团“组成的平面随着时间的变化率等等。\par
用随体导数改写的牛二定理如下:
\begin{equation}
m\frac{D\vec{u}}{Dt}=\vec{F}
\end{equation}
OK,接下来我们就来看下粒子上受到了哪些力。最简单的当然是重力:$m\vec{g}$。但是这肯定不是唯一的力,其他粒子也对这个粒子有作用力。\par
第一种来自其他粒子的力,是压力。高压区域的粒子持续不断的给低压区域的粒子施加压力,比如正是由于有心脏不断跳动,向血液施加压力,血液才能流遍全身)。请注意,我们这里的压力,考虑的是合力,具体这个力是从哪个粒子来的并不重要。大家都挤过地铁吧,想象我们现在站在一节塞满了人的地铁车厢里面,真的是塞得满满当当的。这时候突然你的左手边开门了,瞬间你左边的人都走光了,这时候你会受什么力呢?当然是从右向左的力啦,你右边的人也想下车但是被你挡住了啊,只能推你!\par
具体到流体的粒子,假设流体受到的总压力分布为p(注意,这是一个矢量场,不同的位置有不同的压力值), 那么在某点的压强是多少呢?在该点对压力取梯度就可以了,也就是$-\nabla p$,那么我们这个”分子团“上面受到的总压力是多少呢?理论上讲是对$-\nabla p$在整个粒子团的表面取积分,这么做很难,所以我们一般就简简单单乘以体积进行近似。结果粒子上的总压力为:$-V\nabla p$。负号表示这个力是粒子团抵抗外界压力的力,压力的存在主要是因为粒子团想要维持自己的体积,这一点在后续”不可压缩条件“这部分会有更详细解释。\ par
第二种来自其他粒子的力为粘性力,

\end{document}
