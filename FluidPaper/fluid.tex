\documentclass{article}
\usepackage{ctex}  %加载包,因为我们在用中文写文档,所以必须加载这个包,否则不支持中文
\usepackage{multicol}  %加载包
\usepackage[top=1in, bottom=1in, left=1.25in, right=1.25in]{geometry}  %加载包
\usepackage{lscape}
\usepackage[colorlinks,linkcolor=blue]{hyperref}
\author{VincentZhang}  % 作者
\date{2019.7.30}  %定义时间
\title{流体力学}  %文档标题
\begin{document}

\maketitle
前言,这篇文章是2007年Siggraph的课程文档(Course Note),从头到底很详细的讲解了怎么用计算机模拟流体(烟和水等等)。
原文地址: \url{https://www.cs.ubc.ca/~rbridson/fluidsimulation/fluids_notes.pdf}

\section{第一章 流体力学公式}
流体力学中最重要的公式是Navier-Stokes公式:
\begin{equation}
\frac{\partial{\vec{u}}}{\partial{t}}+\vec{u}\cdot \nabla{\vec{u}}+\frac{1}{\rho}\nabla{p}=\vec{g}+\nu\nabla\cdot\nabla\vec{u} \label{momentumequation}
\end{equation}
\begin{equation}
\nabla\cdot\vec{u}=0 \label{incompressibilitycondition}
\end{equation}

\subsection{符号含义}
其中$\vec{u}$是流体的速度。\par
$\rho$是流体的密度,大家都熟悉,如果是水的话,其值为$1000kg/m^3$. 如果是空气的话,其值为$1.3kg/m^3。$\par
$p$是压力,流体对内对外的压力。\par
$vec{g}$是大家都熟悉的重力加速度,一般性就是中学书上的值$(0,-9.81,0)m/s^2$。从这个向量表示也行,我们这里y轴像上。x-轴,y-轴是水平的。$\vec{g}$有时候大家也会叫他"体作用力"\par
$\nu$是“静态粘度”。顾名思义,这个参数表达的是流体有多粘稠。像糖浆这样的流体这个值很高,像酒精这样的流体这个值很低。学术点说,这个参数反映了流体抗拒形变的程度。一个极端的例子是沥青,网上不是有个段子,有个科学家等了一辈子想看一颗沥青掉下来的过程,最后到死都没有实现。另一个更加极端的例子是玻璃,玻璃其实也是一种流体,有很多几百年老教堂的玻璃窗,下面比上面厚,这就是流体流动的结果,但是大家懂的,要观察玻璃滴下来,估计这个老教授要等到宇宙尽头了。
\subsection{动量方程}
Navier-Stokes里的第一条方程\ref{momentumequation},其实是一条向量方程,另有一个名字叫做"动量方程",所以看这条方程的时候,千万要注意变量上面的箭头。其实这个方程就是变形的牛二方程$\vec{F}=m\vec{a}$。这个方程告诉了我们流体在各种作用力之下是怎么运动的。显然这条方程比较复杂,这一节将把这个方程拆开一项项给大家讲解。后一章节,我们将介绍\ref{incompressibilitycondition},这个方程叫做“不可压缩条件”。


\end{document}
